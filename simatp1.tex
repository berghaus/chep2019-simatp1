\documentclass{webofc}
\usepackage[varg]{txfonts}   % Web of Conferences font
%
% Put here some packages required or/and some personnal commands
%
%
\begin{document}
%
\title{ATLAS Sim@P1 upgrades during long shutdown two}
%
% subtitle is optionnal
%
%%%\subtitle{Do you have a subtitle?\\ If so, write it here}

\author{\firstname{Frank} \lastname{Berghaus}\inst{1}\fnsep\thanks{\email{berghaus}} \and
        \firstname{Franco} \lastname{Brasolin}\inst{2}\fnsep \and
        \firstname{Alessandro} \lastname{Di Girolamo}\inst{3}\fnsep \and
        \firstname{Marcus} \lastname{Ebert}\inst{1}\fnsep \and
        \firstname{Colin Roy} \lastname{Leavett-Brown}\inst{1}\fnsep \and
        \firstname{Chris} \lastname{Lee}\inst{4}\fnsep \and
        \firstname{Peter}\lastname{Love}\inst{5}\fnsep \and
        \firstname{Eukeni} \lastname{Pozo Astigarraga}\inst{3}\fnsep \and
        \firstname{Diana} \lastname{Scannicchio}\inst{6}\fnsep \and
        \firstname{Jaroslava} \lastname{Schovancova}\inst{3}\fnsep \and
        \firstname{Rolf} \lastname{Seuster}\inst{1}\fnsep \and
        \firstname{Randall} \lastname{Sobie}\inst{1}\fnsep\thanks{\email{sobie}}
}

\institute{University of Victoria, Victoria, Canada
\and
           Universita e INFN, Bologna, Italy
\and
           CERN, Geneva, Switzerland
\and
           University of Cape Town, Cape Town, South Africa
\and
           Lancaster University, Lancaster, United Kingdom
\and
           University of California Irvine, Irvine, United States of America
          }

\abstract{%
  The Simulation at Point1 (Sim@P1) project was built in 2013 to take advantage
  of the ATLAS Trigger and Data Acquisition High Level Trigger (HLT) farm. The
  HLT farm provides around 100,000 cores, which are critical to ATLAS during
  data taking. When ATLAS is not recording data, this large compute resource is
  used to generate and process simulation data for the experiment. At the
  beginning of the current long shutdown (LS2), the HLT farm including the
  Sim@P1 infrastructure was upgraded. Previous papers emphasized the need for
  “simple, reliable, and efficient tools” and assessed various options to
  quickly switch between data acquisition operation and offline processing. In
  this contribution we describe the new mechanisms put in place for the
  opportunistic exploitation of the HLT farm for offline processing and give
  results from the first months of operation.
}
%
\maketitle
%
\section{Introduction}
\label{intro}
ATLAS~\cite{atlas} is a general purpose experiment located at point one (P1) of CERN's large hadron collider. ATLAS employs a large computer farm, summarized in table~\ref{tab:hlt_hardware}, to facilitate data acquisition and event selection.
\begin{table}
\centering
\caption{The hardware at P1 currently available for use with Sim@P1. The C6100 nodes are the decommissioned old HLT. They provide 11008 hyper-threaded (HT) cores permanently running
in Sim@P1 mode. Not all cores are used to ensure the virtual machines provide
sufficient memory for ATLAS offline workloads. The other hardware is switched to
Sim@P1 mode when data taking is not foreseen in the next 24 hours. These
opportunistic resources provide up to 97216 additional cores. Usually the trigger
and data acquisition team retains some resources for their needs.}
\label{tab:hlt_hardware}       % Give a unique label
% For LaTeX tables you can use
\begin{tabular}{llllll}
\hline
Product name & Intel\textsuperscript{\textregistered} Xeon\textsuperscript{\textregistered} & HT Cores & Memory [GB] & VM Cores & Nodes \\\hline
C6100 & X5650 & 24 & 24 & 16 & 688\\
Centerprise & E5-2650 v4 & 48 & 64 & 48 & 360 \\
Persy & E5-2660 v4 & 56 & 64 & 56 & 440 \\
MegWare & E5-2680 v3 & 48 & 64 & 48 & 680 \\
QuantaPlex T41S-2U & E5-2680 v3 & 48 & 64 & 48 & 472\\
\end{tabular}
\end{table}
The Sim@P1 project aims to opportunistically use the trigger and data acquisition high level trigger resources for offline computing. The High Level Trigger (HLT)~\cite{tdaq2013} is a mission critical part of ATLAS data taking and is physically connected to the control network of the detector and the ``data'' network which allows connections to the CERN data centre through a switch at P1. When working with Sim@P1 it is important to ensure the secure isolation from the physical
resources at P1, seamless integration into the ATLAS distributed computing
system, and reliable transition between the functions of the resources. A
system satisfying these criteria was developed during the first long shutdown of
the LHC~\cite{Ballestrero:2015ypa}. Isolation is achieved by running virtual
machines on the physical HLT hardware. The virtual machines are managed using the
cloud framework OpenStack~\cite{openstack}. The virtual machines share
the ``data'' connection of the HLT hardware through a tagged VLAN providing
network isolation on the level of the Ethernet frame managed by the switches. This
VLAN allows the Virtual Machines to connect to a controlled list of interfaces in
the CERN general purpose network. This list specifies the interface of the machines needed to allow offline workloads
to be delivered and executed. To minimize impact of Sim@P1 on the Trigger and Data Acquisition (TDAQ) operation, only
simulation tasks from the central production system are submitted to
run at P1.

The original implementation of Sim@P1 ran successfully during the first long
shutdown of the LHC facilities between 2013 and 2015. Once the experiment resumed
data-taking, the system was used in
opportunistic mode~\cite{Ballestrero:2017psv}. The HLT was switched from TDAQ
function to Sim@P1 mode
for intervals of a few days during technical stops and machine development. To
allow this opportunistic usage a set of scripts were developed to manage the
transition of resources between TDAQ and Sim@P1 mode.

During the second long shutdown of the LHC facilities, starting in 2019, the HLT
will be upgraded. Along with the upgrades to the HLT we plan to upgrade the Sim@P1
infrastructure. We plan to replace the controller nodes, upgrade to a new version
of OpenStack, and replace the scripts managing the virtual machines with
Cloudscheduler~\cite{Sobie:2013zua}.

\section{Section title}
\label{sec-1}
For bibliography use \cite{RefJ}
\subsection{Subsection title}
\label{sec-2}
Don't forget to give each section, subsection, subsubsection, and
paragraph a unique label (see Sect.~\ref{sec-1}).

For one-column wide figures use syntax of figure~\ref{fig-1}
\begin{figure}[h]
% Use the relevant command for your figure-insertion program
% to insert the figure file.
\centering
\includegraphics[width=1cm,clip]{tiger}
\caption{Please write your figure caption here}
\label{fig-1}       % Give a unique label
\end{figure}

For two-column wide figures use syntax of figure~\ref{fig-2}
\begin{figure*}
\centering
% Use the relevant command for your figure-insertion program
% to insert the figure file. See example above.
% If not, use
\vspace*{5cm}       % Give the correct figure height in cm
\caption{Please write your figure caption here}
\label{fig-2}       % Give a unique label
\end{figure*}

For figure with sidecaption legend use syntax of figure
\begin{figure}
% Use the relevant command for your figure-insertion program
% to insert the figure file.
\centering
\sidecaption
\includegraphics[width=5cm,clip]{tiger}
\caption{Please write your figure caption here}
\label{fig-3}       % Give a unique label
\end{figure}

For tables use syntax in table~\ref{tab-1}.
\begin{table}
\centering
\caption{Please write your table caption here}
\label{tab-1}       % Give a unique label
% For LaTeX tables you can use
\begin{tabular}{lll}
\hline
first & second & third  \\\hline
number & number & number \\
number & number & number \\\hline
\end{tabular}
% Or use
\vspace*{5cm}  % with the correct table height
\end{table}
%
% BibTeX or Biber users please use (the style is already called in the class, ensure that the "woc.bst" style is in your local directory)
% \bibliography{name or your bibliography database}
%
% Non-BibTeX users please use
%
\begin{thebibliography}{}
  \bibitem{atlas}
   The ATLAS Collaboration
   %The  ATLAS  Experiment  at  the  CERN  Large  Hadron  Collider.
   JINST \textbf{3} S08003 (2008)

  \bibitem{tdaq2013}
  The ATLAS Collaboration, \textit{Technical Design Report for the Phase-I Upgrade of the ATLAS TDAQ System} (CERN, Geneva, 2013) 120-122

  %\cite{Ballestrero:2015ypa}
  \bibitem{Ballestrero:2015ypa}
    S.~Ballestrero {\it et al.}
    %``Design, Results, Evolution and Status of the ATLAS Simulation at Point1 Project,''
    J.\ Phys.\ Conf.\ Ser.\  {\bf 664} 2  022008 (2015)
    %doi:10.1088/1742-6596/664/2/022008
    %%CITATION = doi:10.1088/1742-6596/664/2/022008;%%

  \bibitem{openstack}
  Openstack project, ”OpenStack” [software], version Icehouse, available from\linebreak
  https://www.openstack.org/software/icehouse/ [accessed 2018-09-24]

  %\cite{Ballestrero:2017psv}
  \bibitem{Ballestrero:2017psv}
    S.~Ballestrero {\it et al.}
    %``Evolution and experience with the ATLAS Simulation at Point1 Project,''
    J.\ Phys.\ Conf.\ Ser.\  {\bf 898}  8 082012 (2017)
    %doi:10.1088/1742-6596/898/8/082012
    %%CITATION = doi:10.1088/1742-6596/898/8/082012;%%

  %\cite{Sobie:2013zua}
  \bibitem{Sobie:2013zua}
    R.~J.~Sobie {\it et al.}
    %``HTC Scientific Computing in a Distributed Cloud Environment,''
    arXiv:1302.1939 [cs.DC]
    %%CITATION = ARXIV:1302.1939;%%
%
% and use \bibitem to create references.
%
\bibitem{RefJ}
% Format for Journal Reference
Journal Author, Journal \textbf{Volume}, page numbers (year)
% Format for books
\bibitem{RefB}
Book Author, \textit{Book title} (Publisher, place, year) page numbers
% etc
\end{thebibliography}

\end{document}

% end of file template.tex

<div id='footer'><table width='100%'><tr><td class='right'><a href='http://fusioninventory.org/'><span class='copyright'>FusionInventory 9.1+1.0 | copyleft <img src='/glpi/plugins/fusioninventory/pics/copyleft.png'/>  2010-2016 by FusionInventory Team</span></a></td></tr></table></div>
